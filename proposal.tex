\documentclass{article}

\title{Determining the Complexity of Parallel Circuits: A Proposal}
\date{May 2017}
\author{Brian Lee}
\usepackage{circuitikz}
\usepackage{wrapfig}
\usepackage[margin=1.0in]{geometry}
\begin{document}

\maketitle

\begin{abstract}
%.. add detailed explanation of what constitutes 'P' - completeness
The goal of the experiment is to determine if Parallel Circuits are $NP$-Complete or $P$-Complete. For this purpose, parallel circuits will be augmented with switches, which will be modelled in a computer program's logic flow. To start, three parallel circuits are to be constructed, consisting of 1,2, and 3 resistors respectively. From there, induction will be utilized to attempt to prove that the number of resistors can go up to infinity. The experiment is a form of inquiry into the prospect of modelling circuits using computers, which has numerous potential applications. It will also attempt to speculate, depending on the results, if $P=NP$.
\end{abstract}

\section{Introduction}
\subsection{Parallel Circuits}
A \textbf{Parallel Circuit} is a variation of circuit that, in its most elemental version, separates $I_{net}$ into separate branches, dedicating a resistor for each branch. The general idea behind a parallel circuit is that each resistor recieves an independent supply of $I$, such that if one resistor loses $I$, the other resistors maintain $I$. The following relationships are derivable using Olm's Law ($V=IR$):\\ %.. needs citation
\begin{center}
\begin{minipage}[c]{0.5\textwidth}
$$V_{net} = V_{1} = V_{2} = ... = V_{n}$$
$$I_{net} = \Sigma I_{n}$$
$$\frac{1}{R_{net}} = \Sigma \frac{1}{R_{n}}$$
\end{minipage}
\begin{minipage}[c]{0.4\textwidth}
\begin{circuitikz} %.. need to add caption here
	\draw
	(0,0) to[battery] (0,2) -- (3,2)
	to[R] (3,0) -- (0,0);
	\draw
	(1,2) to[R] (1,0);
	\draw
	(2,2) to[R] (2,0);
\end{circuitikz}
\end{minipage}
\end{center}
Several unique properties emerge if $V$, $I$, and $R$ values are tracked over changes made to the circuit, such as removing/adding a resistor, changing $V$, etc. Those systematic changes will become central to the experiment. %.. insert citation
\begin{thebibliography}{9}

\end{thebibliography}

\end{document}

