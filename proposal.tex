\documentclass{article}

\title{Determining the Complexity of Parallel Circuits: A Proposal}
\date{May 2017}
\author{Brian Lee}
\usepackage{circuitikz}
\usepackage{wrapfig}
\usepackage[margin=1.0in]{geometry}
\usepackage{listings}
\begin{document}

\maketitle

\begin{abstract}
%.. add detailed explanation of what constitutes 'P' - completeness
The goal of the experiment is to determine if Parallel Circuits are $NP$-Complete or $P$-Complete. For this purpose, parallel circuits will be augmented with switches, which will be modelled in a computer program's logic flow. To start, three parallel circuits are to be constructed, consisting of 1,2, and 3 resistors respectively. From there, induction will be utilized to attempt to prove that the number of resistors can go up to infinity. The experiment is a form of inquiry into the prospect of modelling circuits using computers, which has numerous potential applications. It will also attempt to speculate, depending on the results, if $P=NP$.
\end{abstract}

\section{Introduction}
Given that the experiment incorporates some higher-level concepts within Computer Science, it's appropriate to outline them here for a proper grasp of the experiment's goals/function. Please note, however, that the following is a low-level overview meant for a basic understanding. The final paper will approach Computer Science on a substantially more formal level. 
\subsection{Computational Complexity Theory}
Computational Complexity Theory is a branch of Theoretical Computer Science that asks the question of \textit{how hard} certain problems are (in terms of efficiency in solvency). There are two general classes of problems: the first consists of problems that \textit{are} easily processed (by a computer), and has a set runtime of $O(n^{k})$ (in other words, a static polynomial function). Problems here reside within the complexity class \textbf{P} (Polynomial Time), and are described as 'P-Complete.'\cite{michiel}

For example, Consider a computer program that compares an input value with each number of an array of numbers.\footnote{Code is available in the appendix} It would be classified as $P-Complete$ as it has a set runtime of $O(n)$ (the program goes through all the numbers in the array). Almost all the software a typical person uses is composed of $P-Complete$ altorithims (e.g. Microsoft Office).

The other complexity class composes of problems that are \textit{hard} for a computer to solve. There is no distinct $O(n^{k})$. Rather, a computer can \textit{check} solutions to the problem in $P$-time. Problems of this class are said to be $NP$-Complete, or 'Non-Deterministic Polynomial Time.' %.. insert deterministic explanation

A good example of a $NP$-Complete problem is Minesweeper. This conceptually fits: a computer won't be able to exactly model problems based on \textit{random chance} (such as the mines' locations); there will always be scenarios where the algorithim is wrong.\cite{mine} This further implies that computers can't model \textit{everything}.
\subsection{Physics}

\section{Appendix} 
\lstinputlisting[language=Python, caption = Sequential Search]{seqsearch.py}

\medskip

\bibliographystyle{unsrt}
\bibliography{citations}



\end{document}

